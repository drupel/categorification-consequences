\documentclass[12pt]{amsart}
% this is here to force arXiv to produce a nice output
\pdfoutput=1

% Commands for marginal notes below
\usepackage[draft]{say}
\newcommand{\sayD}[1]{\say[D]{#1}}
\newcommand{\sayS}[1]{\say[S]{#1}}
\newcommand{\sayH}[1]{\say[H]{#1}}

\usepackage{mathtools}
\usepackage{amsmath}
\usepackage{amsthm}
\usepackage{amssymb}
\usepackage{amsbsy}
\usepackage{amstext}
\usepackage{amsopn}
\usepackage{enumerate}
\usepackage{xcolor}
\usepackage{graphicx}
\usepackage{microtype}
\usepackage{verbatim}
\usepackage[margin=1in,marginparwidth=0.8in, marginparsep=0.1in]{geometry}
\renewcommand{\baselinestretch}{1.2} % changes page formatting
\usepackage[pagebackref, bookmarks=true, bookmarksopen=true, bookmarksdepth=3,bookmarksopenlevel=2, colorlinks=true, linkcolor=blue, citecolor=blue, filecolor=blue, menucolor=blue, urlcolor=blue]{hyperref}
\usepackage{newtxtext} % changes font appearance, replaces times
\usepackage{stmaryrd}
\usepackage{accents}
\usepackage{bbm}
\usepackage{tikz}

\newtheorem{theorem}{Theorem}
\newtheorem{corollary}[theorem]{Corollary}
\newtheorem{conjecture}[theorem]{Conjecture}
\newtheorem{proposition}[theorem]{Proposition}
\newtheorem{remark}[theorem]{Remark}

\newcommand{\bfa}{\mathbf{a}}
\newcommand{\bfb}{\mathbf{b}}
\newcommand{\bfd}{\mathbf{d}}
\newcommand{\bfe}{\mathbf{e}}
\newcommand{\bfg}{\mathbf{g}}
\newcommand{\bfv}{\mathbf{v}}
\newcommand{\bfw}{\mathbf{w}}
\newcommand{\bfx}{\mathbf{x}}
\newcommand{\bfX}{\mathbf{X}}
\newcommand{\cA}{\mathcal{A}}
\newcommand{\cF}{\mathcal{F}}
\newcommand{\cQ}{\mathcal{Q}}
\newcommand{\cT}{\mathcal{T}}
\newcommand{\FF}{\mathbb{F}}
\newcommand{\diag}{\operatorname{diag}}
\newcommand{\Ext}{\operatorname{Ext}}
\newcommand{\Gr}{\operatorname{Gr}}
\newcommand{\half}{{\frac{1}{2}}}
\newcommand{\Hom}{\operatorname{Hom}}
\newcommand{\rep}{\operatorname{rep}}
\newcommand{\ZZ}{\mathbb{Z}}

%\renewcommand{\thesubsection}{\arabic{subsection}}
%\makeatletter
%\def\@seccntformat#1{\@ifundefined{#1@cntformat}%
%   {\csname the#1\endcsname\quad}%       default
%   {\csname #1@cntformat\endcsname}}%    enable individual control
%\newcommand\section@cntformat{}
%\makeatother

\newenvironment{enumeratea}{\begin{enumerate}[\upshape (a)]}{\end{enumerate}}

\title{Some consequences of categorification}
\author[Rupel]{Dylan Rupel}
\address[Dylan Rupel]{
University of Notre Dame\newline
Department of Mathematics\newline
Notre Dame, IN 46556\newline
USA}
\email{drupel@nd.edu}

\author[Stella]{Salvatore Stella}
\address[Salvatore Stella]{
University of Haifa\newline
Departments of Mathematics and Computer Sciences\newline
Haifa, Mount Carmel, 31905\newline
Israel
}
\email{stella@math.haifa.ac.il}

\begin{document}
\begin{abstract}
  Several conjectures on acyclic skew-symmetrizable cluster algebras are proven as direct consequences of their categorification via valued quivers.
  These include conjectures of Fomin-Zelevinsky, Reading-Speyer, and Reading-Stella related to $\bfd$-vectors, $\bfg$-vectors, and $F$-polynomials.
\end{abstract}
\maketitle

  \section{Introduction}
  The categorification of skew-symmetric cluster algebras using representations of quivers was initiated by Marsh, Reineke, and Zelevinsky \cite{marsh-reineke-zelevinsky} for Dynkin quivers.  
  With the advent of cluster characters \cite{caldero-chapoton}, the subject has exploded as an industry all its own, leading to the publication of numerous articles including \cite{caldero-chapoton-schiffler,buan-marsh-reineke-reiten-todorov,derksen-weyman-zelevinsky,geiss-leclerc-schroer,caldero-keller,caldero-keller2,plamondon,palu,rupel1,qin,rupel2} just to name a few.  

  The main idea is to interpret the combinatorics of cluster mutations in terms of a category of representations of a quiver, and in particular to obtain an interpretation of the initial cluster Laurent expansions of non-initial cluster variables as generating functions for certain geometric invariants (e.g. Euler characteristics, point counts over finite fields, Poincar\'e polynomials, etc.) of varieties of subrepresentations in a given quiver representation.
  
  In the case of skew-symmetric cluster algebras, the rich structure present in the cluster characters defining the categorification has led to proofs of many conjectures relating to the structure of cluster algebras \cite{cerulliirelli-keller-labardinifragoso-plamondon} and the resolution of conjectures describing the internal structure of individual cluster variables as well as providing a tool which governs the combinatorics of mutations.
  The main goal of this note is to observe that the categorification of acyclic skew-symmetrizable (quantum) cluster algebras \cite{rupel1,rupel2} naturally leads to proofs of many of the same conjectures as well as providing proofs of some conjectures not obtainable from the standard quiver (with potential) approach.  

  Some of these are already proven in \cite{demonet}...

  \section{Cluster  Algebras and Quantum Cluster Algebras} Let $\cA(\bfx,\tilde B)$ be an acyclic cluster algebra \cite{berenstein-fomin-zelevinsky}, i.e. the principal submatrix of $\tilde B$ is acyclic.  
  When the exchange matrix $\tilde B$ has full rank, the algebra $\cA(\bfx,\tilde B)$ admits a Poisson structure compatible with mutations \cite{gekhtman-shapiro-vainshtein} and one may obtain a quantum cluster algebra via the naive quantization.  
  More intuitively, for an indeterminate $q$, the definition of the quantum cluster algebra $\cA_q(\bfX,\tilde B)$ can be obtained from that of $\cA(\bfx,\tilde B)$ via the following modifications:
  \begin{itemize}
    \item The initial cluster $\bfX$ consists of $q$-commuting variables, i.e. there exists a skew-symmetric matrix $\Lambda=(\lambda_{ij})$ so that $X_iX_j=q^{\lambda_{ij}}X_jX_i$.  
    Denote by $\cT_\Lambda$ the quantum torus algebra generated by $X_i^{\pm1}$, $1\le i\le m$, over the ring $\ZZ[q^{\pm\half}]$.
    \item To decide how to incorporate $q$ into the exchange relations, we note that the algebra $\cT$ admits an anti-involution (called the \emph{bar-involution}) fixing each $X_i$ and interchanging $q$ with $q^{-1}$.
    Every non-initial cluster variable should also be ``bar-invariant'' and this uniquely determines the power of $q^\half$ by which to multiply each exchange monomial.  
    More precisely, using the bar-invariant monomial basis $X^\bfa$, $\bfa\in\ZZ^m$, of $\cT$ given by
    \[X^\bfa=q^{-\half\sum\limits_{i<j}\lambda_{ij}a_ia_j}X_1^{a_1}\cdots X_m^{a_m},\]
    we may write the exchange relation as $X'_k=X^{\bfb_+^k-\varepsilon_k}+X^{\bfb_-^k-\varepsilon_k}$, where $\varepsilon_k$ denotes the standard basis vector of $\ZZ^m$ and the $k^{th}$ column of $\tilde B$ decomposes as $\bfb^k=\bfb^k_+-\bfb^k_-$ for minimal positive vectors $\bfb^k_+,\bfb^k_-\in\ZZ_{\ge0}^m$.
    \item Each cluster $\bfX'$ obtainable from $\bfX$ by a sequence of mutations should again generate a quantum torus, i.e. consist of $q$-commuting variables, and this forces a compatibility condition (see below) of the commutation matrix $\Lambda$ with the exchange matrix $\tilde B$.
    This compatibility naturally reproduces under mutations.
  \end{itemize}
  Note that the final compatibility condition is identical to the conditions necessary for a compatible Poisson structure.
  
  By work of Berenstein and Zelevinsky \cite{berenstein-zelevinsky}, the famous Laurent phenomenon holds in this adapted setting and one may ask how to describe the initial cluster Laurent expansions of non-initial quantum cluster variables.  
  The answer will be given using the representation theory of valued quivers in the following section.

  Following work of Tran \cite{tran}, it suffices to consider the principal coefficients case defined as follows.  
  Consider an $n\times n$ skew-symmetrizable matrix $B$ and write $\tilde B=\left[\begin{array}{c}B\\ I_n\end{array}\right]$ for the corresponding principalized exchange matrix.  
  Fix a choice of diagonal symmetrizing matrix $D=\diag(d_1,\ldots,d_n)$ with $d_i\in\ZZ_{>0}$, i.e. such that $DB$ is skew-symmetric.
  Let $\Lambda$ be a skew-symmetric $2n\times2n$ matrix compatible with $\tilde B$, e.g. given any $n\times n$ skew-symmetric matrix $\Lambda_0$ we may take $\Lambda=\left[\begin{array}{cc}\Lambda_0 & -\Lambda_0B-D\\ -B^T\Lambda_0+D & B^T\Lambda_0B+B^TD\end{array}\right]$ so that $\tilde B^T\Lambda=\big[\,D\ \boldsymbol{0}\,\big]$ (this is the required compatibility condition).

  \section{Quantum Cluster Characters}
  Define a quiver $Q_B$ with vertices $\{1,\ldots,n\}$ and $\gcd(b_{ij},b_{ji})$ arrows $i\longrightarrow j$ whenever $b_{ij}<0$, we assume this quiver to be acyclic.
  Write $\tilde Q$ for the quiver obtained from $Q_B$ by attaching to each vertex $i$ an additional vertex $n+i$ via a single arrow $i\to n+i$.  
  The Laurent expansions of non-initial quantum cluster variables can be described in terms of the representation theory of the valued quiver $(\tilde Q,\tilde D)$ where $\tilde D=\left[\begin{array}{cc}D&0\\0&D\end{array}\right]$.

  Let $\FF$ be a finite field and choose an algebraic closure $\overline{\FF}$ of $\FF$.  
  Write $\FF^{\langle d\rangle}$ for the degree $d$ extension of $\FF$ inside $\overline{\FF}$.
  Note that this provides a canonical identification of $\FF^{\langle d\rangle}$ as a vector space over $\FF^{\langle g\rangle}$ whenever $g|d$.  
  A representation $V=(V_i,V_a)$ of $(Q,D)$ consists of an $\FF^{\langle d_i\rangle}$-vector space $V_i$ for each vertex $i$ and an $\FF^{\langle\gcd(d_i,d_j)\rangle}$-linear map $V_a:V_i\to V_j$ for each arrow $a:i\to j$ (i.e. when $b_{ij}<0$).  
  The finite-dimensional representations of $(Q,D)$ form a hereditary abelian category denoted by $\rep_\FF(Q,D)$.  
  The category $\rep_\FF(Q,D)$ naturally embeds into the category $\rep_\FF(\tilde Q,\tilde D)$, it is this subcategory that we will be interested in and thus always consider $\rep_\FF(Q,D)$ to be this subcategory of $\rep_\FF(\tilde Q,\tilde D)$.  


  Write $\tilde\cQ$ for the Grothendieck group of $\rep_\FF(\tilde Q,\tilde D)$ and let $\cQ\subset\tilde\cQ$ denote the Grothendieck group of $\rep_\FF(Q,D)$.
  Since $\tilde Q$ is acyclic, there is an identification $\tilde\cQ\cong\ZZ^{2n}$ by taking classes $\alpha_i=[S_i]$ of the vertex-simple representations as a basis and the Euler-Ringel form $\langle V,W\rangle:=\dim_\FF\Hom(V,W)-\dim_\FF\Ext(V,W)$ only depends on the classes of $V$ and $W$ in $\tilde\cQ$.  
  More precisely, the Euler-Ringel form may be given in the vertex-simple basis by
  \[\langle\alpha_i,\alpha_j\rangle=\begin{cases} d_i & \text{if $i=j$;}\\d_ib_{ij} & \text{if $b_{ij}<0$;}\\-d_i & \text{if $j=n+i$;}\\0 & \text{otherwise.}\end{cases}\]
  For $\bfe\in\cQ$, define ${}^*\bfe=\sum\limits_{i=1}^n\frac{1}{d_i}\langle\alpha_i,\bfe\rangle\cdot\alpha_i\in\cQ$ and $\bfe^*=\sum\limits_{j=1}^{2n}\frac{1}{d_j}\langle\bfe,\alpha_j\rangle\cdot\alpha_j\in\tilde\cQ$, noting that these do not depend on the choice of symmetrizing matrix $D$. 
  Then the \emph{quantum cluster character} of a representation $V\in\rep_\FF(Q,D)$ with $[V]=:\bfv$ is given by \[X_V=\sum\limits_{\bfe\in\cQ} |\FF|^{-\half\langle\bfe,\bfv-\bfe\rangle}\big|\!\Gr_\bfe(V)\big|X^{\tilde B\bfe-{}^*\bfv}\in\cT_\Lambda,\] where $\Gr_\bfe(V)$ denotes the set of all subrepresentations $E\subset V$ with isomorphism class $[E]=\bfe$.  
  Note that $\tilde B\bfe={}^*\bfe-\bfe^*$ and for $1\le j\le n$ we have ${}^*\alpha_j=\alpha_j+\sum\limits_{i:b_{ij}<0}b_{ij}\alpha_i$.

  A representation $V$ is called \emph{rigid} if $\Ext(V,V)=0$.  
  \begin{theorem}\cite{rupel2}
    \label{th:quantum cluster characters}\mbox{}
    \begin{enumeratea}
      \item For each rigid indecomposable $V\in\rep_\FF(Q,D)$, the quantum cluster character $X_V$ is a quantum cluster variable of $\cA_{|\FF|}(\bfX,\tilde B)$.  
      Moreover, every non-initial cluster variable of $\cA_{|\FF|}(\bfX,\tilde B)$ arises in this way.
      \item For each $\bfe\in\cQ$ and each positive root $\bfv\in\cQ$, there exists a polynomial $P_{\bfv,\bfe}(q)$ such that for any rigid indecomposable $V\in\rep_\FF(Q,D)$ with isomorphism class $\bfv$, we have $\big|Gr_\bfe(V)\big|=P_{\bfv,\bfe}\big(|\FF|\big)$.  These polynomials give a ``generic'' quantum cluster character 
      \[X_\bfv=\sum\limits_{\bfe\in\cQ} q^{-\half\langle\bfe,\bfv-\bfe\rangle}P_{\bfv,\bfe}(q)X^{\tilde B\bfe-{}^*\bfv}\]
      which computes the cluster variables of $\cA_q(\bfX,\tilde B)$ with arbitrary parameter $q$.
    \end{enumeratea}
  \end{theorem}
  An analogous result was obtained for acyclic skew-symmetric quantum cluster algebras by Qin \cite{qin} using representations of acyclic quivers and with counting polynomials replaced by Poincar\'e polynomials.  Setting $q=1$ in Theorem~\ref{th:quantum cluster characters}(b) gives the following result from which we will deduce the main results of this note.
  \begin{corollary}
    \label{cor:classical cluster characters}
    All non-initial cluster variables of the cluster algebra $\cA(\tilde B)$ are computed by the cluster characters
    \[x_\bfv=x^{-{}^*\bfv}\sum\limits_{\bfe\in\cQ} P_{\bfv,\bfe}(1)x^{\tilde B\bfe}\]
    as $\bfv$ ranges over all positive roots in the root system of the Cartan companion of $B$.
    In particular, the $g$-vector is given by $-{}^*\bfv$ and the $F$-polynomial is $\sum\limits_{\bfe\in\cQ} P_{\bfv,\bfe}(1)y^\bfe$.
    %Moreover, $F$-polynomials have constant term 1 and all coefficients are positive.
  \end{corollary}
  \begin{proof}
    %The final claim follows from the results of \cite{wolf} (not for valued quivers).
  \end{proof}
  \begin{remark}
    It immediately follows that the $F$-polynomials have constant term 1 since $P_{\bfv,0}\equiv1$ for any $\bfv\in\cQ$.
    This proves \cite[Conj. 5.4]{fomin-zelevinsky4}.
    It immediately follows also that \cite[Conj. 5.5]{fomin-zelevinsky4} holds for acyclic initial seeds.
  \end{remark}

  The connection between the representation theory of $(Q,D)$ and the cluster algebra is actually much stronger.  For a source (resp. sink) vertex $k$ in $Q$, write $\Sigma_k^-:\rep_\FF(Q,D)\to\rep_\FF(\mu_kQ,D)$ (resp. $\Sigma_k^+:\rep_\FF(Q,D)\to\rep_\FF(\mu_kQ,D)$) for the reflection functor as defined in \cite[Sec. 2]{dlab-ringel}.  
  In what follows, we will drop the adornment and simply write $\Sigma_k$ for both reflection functors, which functor to apply should be clear from context.
  For $1\le k\le n$, write $\rep_\FF^{\langle k\rangle}(Q,D)\subset\rep_\FF(Q,D)$ for the full subcategory consisting of representations which contain no summands isomorphic to the simple representation $S_k$.
  \begin{theorem}\cite{rupel1}
    \label{th:reflection functor}
    Let $k$ be a sink or a source in $Q$ and write $(\bfX',\tilde B')=\mu_k(\bfX,\tilde B)$.  
    For any representation $V\in\rep_\FF^{\langle k\rangle}(Q,D)$ we have $X_V=X'_{\Sigma_kV}$, where $X'_{\Sigma_kV}$ denotes the quantum cluster character of $\Sigma_kV\in\rep_\FF(\mu_kQ,D)$ in the variables $\bfX'$.
  \end{theorem}

  \section{Deducing the Conjectures}
  Introduce $g$-vectors and $F$-polynomials?

  \begin{proposition}
    \label{prop:principal F-polynomials}
    Let $B$ be an $n\times n$ acyclic exchange matrix.
    Assume $k$ is a sink for $B$ and write $B'=\mu_k B$ for the mutation in direction $k$.
    Consider the tropical evaluation of the $F$-polynomials for $(B')_{prin}$ at the frozen variables corresponding to $c$-vectors of $\mu_k B_{prin}$.
    This evaluation gives the value 1 for all $F$-polynomials except for the one obtained by mutating in direction $k$.
  \end{proposition}
  \begin{proof}
    The terms of the $F$-polynomial are labeled by subrepresentations of the given rigid representation and the claim is that each of these monomials produces only non-negative exponents when evaluated at the given frozen variables.
    By definition of the $c$-vectors for $\mu_k B_{prin}$, it is only the exponent of $y_k$ that could end up being negative.
    However, observe that the total exponent is given by applying the simple reflection $s_k$ to the dimension vector of the subrepresentation.
    Since vertex 1 is a source, the simple $S_k$ cannot be a summand of any subrepresentation (assuming we were not talking about $x_{k;t_0}$).
    It follows that applying the simple reflection on dimension vectors corresponds to applying the corresponding reflection functor on quiver representations, in particular the reflected dimension vector is still a dimension vector and thus a negative exponent will never appear.
    This gives the claim.
  \end{proof}

  The following, in particular, proves \cite[Conj. 1.9]{reading-stella} in the case of an acyclic initial exchange matrix.
   \begin{proposition}
    \label{prop:denominators}
    The denominator vector of $X_V$ is precisely the dimension vector $\bfv$ of $V$.
  \end{proposition}
  \begin{proof}
    The proof is identical to that of \cite[Sec. 4, Cor. 2]{caldero-keller} with appropriate modifications in the valued quiver setting, we recall the details here for convenience of the reader.  

    First note that for any representation $W$ with $[W]=\bfw$ we have $\frac{1}{d_i}\langle W,I_i\rangle=w_i=\frac{1}{d_i}\langle P_i,W\rangle$ where $I_i$ and $P_i$ denote respectively the injective hull and projective cover of the vertex simple $S_i$.  Next using the injective resolution of $S_i$ we see that $\langle W,S_i\rangle\le\langle W,I_i\rangle$ while using the projective resolution gives $\langle S_i,W\rangle\le\langle P_i,W\rangle$.  Now consider a subrepresentation $E\subset V$ with $[E]=\bfe$, applying the above considerations we see that the $i^{th}$ component of $\bfe^*+{}^*(\bfv-\bfe)$ is bounded by the $i^{th}$ component of $\bfv$:
    \[\frac{1}{d_i}\langle E,S_i\rangle+\frac{1}{d_i}\langle S_i,V/E\rangle\le\frac{1}{d_i}\langle E,I_i\rangle+\frac{1}{d_i}\langle P_i,V/E\rangle=e_i+(v_i-e_i)=v_i.\]

    To finish the proof, for each $1\le i\le n$ we must exhibit a subrepresentation $E\subset V$ which realizes this bound.  
    To construct such a subrepresentation, let $J_i$ be the set of all vertices $j$ in $Q$ for which there exists a path (possibly trivial) beginning at $i$ and ending at $j$.  
    Now set $E_j=V_j$ for $j\in J_i$ and $E_j=0$ for $j\notin J_i$.  
    Recall that in an injective resolution $0\longrightarrow S_i\longrightarrow I_i\longrightarrow I\longrightarrow 0$ the injective representation $I$ has nonzero components only at vertices $j$ which admit a nontrivial path \emph{to} vertex $i$, while in a projective resolution $0\longrightarrow P\longrightarrow P_i\longrightarrow S_i\longrightarrow 0$ the projective representation $P$ has nonzero components only at vertices $j$ which admit a nontrivial path \emph{from} vertex $i$.  
    It follows that $\langle E,S_i\rangle=\langle E,I_i\rangle-\langle E,I\rangle=\langle E,I_i\rangle$ and $\langle S_i,V/E\rangle=\langle P_i,V/E\rangle-\langle P,V/E\rangle=\langle P_i,V/E\rangle$ and thus 
    \[\frac{1}{d_i}\langle E,S_i\rangle+\frac{1}{d_i}\langle S_i,V/E\rangle=\frac{1}{d_i}\langle E,I_i\rangle+\frac{1}{d_i}\langle P_i,V/E\rangle=e_i+(v_i-e_i)=v_i\]
    as desired.
  \end{proof}

  With this we obtain a proof of \cite[Conj. 1.8]{reading-stella} and hence, following \cite[Prop. 1.10]{reading-stella}, a proof of \cite[Conj. 1.7]{reading-stella}.
  \sayS{Check the references here, something is fishy}
  \begin{corollary}
    The mutation of the initial cluster at a sink or a source vertex transforms denominator vectors according to the simple reflection associated to that vertex.
  \end{corollary}
  \begin{proof}
    By Theorem~\ref{th:reflection functor}, the initial cluster mutation at a sink or a source vertex transforms quantum cluster characters according to the associated reflection functor on $\rep_{\FF_q}(Q,D)$.  The result then follows from \cite[Prop. 2.1]{dlab-ringel}.
  \end{proof}

  Let $E=(e_{ij})$ be the matrix given by
  \[e_{ij}=\begin{cases} -1 & \text{if $i=j$;}\\ [-b_{ij}]_+ & \text{if $i\ne j$.}\end{cases}\]
  The following result establishes \cite[Conj. 3.21]{reading-speyer}.
  \begin{theorem}
    \label{th:d to g}
    Let $\bfd_i^t\in\ZZ^n$ be a denominator vector for a cluster variable of $\cA(\bfX,\tilde B)$ and write $\bfg_i^t$ for the $g$-vector of that cluster variable.  Then $\bfg_i^t=E\bfd_i^t$.
  \end{theorem}
  \begin{proof}
    Since the $g$-vector of the cluster variable $x_\bfv$ is given by $-{}^*\bfv$, the result follows immediately from the definitions of $E$ and the operator ${}^*(-)$.
  \end{proof}

  Here are some (possible) further results:
  \begin{itemize}
    \item Does \cite[Conj. 6.13]{fomin-zelevinsky4} follow from Theorem~\ref{th:d to g} in the acyclic case?
    \item \cite[Conj. 7.17]{fomin-zelevinsky4} is immediate from Corollary~\ref{cor:classical cluster characters}, this immediately implies \cite[Conj. 6.11]{fomin-zelevinsky4} using \cite[Prop. 7.16]{fomin-zelevinsky4}.  
      Thus we obtain the somewhat surprising fact that the $F$-polynomial determines the $g$-vector for acyclic cluster algebras (of full rank?).
    \item It seems that \cite[Con. 7.6]{fomin-zelevinsky4}/\cite[Conj. 3.22]{reading-speyer} follows from known representation theoretic statements.
      This then implies \cite[Conj. 7.10]{fomin-zelevinsky4} using Theorem~\ref{th:d to g}.
      Then following \cite[Rem. 7.11]{fomin-zelevinsky4}, we obtain \cite[Con. 7.2]{fomin-zelevinsky4}.
    \item Does any part of \cite[Conj. 7.4]{fomin-zelevinsky4} follow?  
      Probably, maybe use Assem-Schifler-Schramchenko.
      Rigidity of cluster tilting objects together with Proposition~\ref{prop:denominators} should prove part 2 and possibly part 3.
    \item It seems like \cite[Conj. 3.9]{reading-speyer} should follow from known tilting theory.
    \item Other conjectures from \cite{reading-speyer}?
  \end{itemize}


% bibliography
\bibliographystyle{amsalpha}
\bibliography{bibliography}



\section{Conjectures in the wild}

\begin{conjecture}
  Cluster variables are Laurent polynomials with positive coefficients.
\end{conjecture}
\begin{proof}
  I am not sure this is really a conjecture: it misses the ``when expanded in every cluster'' part.
  In any case this is obvious in the acyclic initial cluster we fixed because the coefficients in the quantum cluster character are cardinalities.
\end{proof}

\begin{conjecture}[{\cite[Conjecture 4.14(1)]{FZ03}}]
  The exchange graph of a cluster algebra depends only on its exchange matrix and not on choice of coefficients.
\end{conjecture}
\begin{proof}
  I am not sure this follows from categorification.
\end{proof}

\begin{conjecture}[{\cite[Conjecture 4.7]{fomin-zelevinsky4}}]
  In a cluster algebra with principal coefficients any cluster is uniquely determined by its exchange matrix and $c$-vectors.
\end{conjecture}
\begin{proof}
  By \cite{NZ12} $g$-vectors are the the dual basis of the Langland's dual cluster algebra (which also has acylic initial seed). 
  Then use the fact that cluster monomials are parametrized by $g$-vectors.
\end{proof}

\begin{conjecture}[{\cite[Conjecture 4.8]{fomin-zelevinsky4}}]
  The cluster algebra associated to an exchange matrix $B$ is of finite type if and only if the mutation class of the principal extension of $B$ is finite.
\end{conjecture}
\begin{proof}
  The ``if'' part is the only one still open and I do not think we can say anything sensible here
\end{proof}

\begin{conjecture}[{\cite[Conjecture 5.4]{fomin-zelevinsky4}}]
  $F$-polynomials have constant term 1.
\end{conjecture}
\begin{proof}
  The constant term is given by the empty subrepresentation.
\end{proof}
\begin{corollary}
  By \cite[Proposition 5.6]{fomin-zelevinsky4} $c$-vectors are sign-coherent.
\end{corollary}
\begin{corollary}
  This implies most of the conjectures we care about: see \cite[Proposition 4.2]{NZ12}.
\end{corollary}

\begin{conjecture}[{\cite[Conjecture 5.5]{fomin-zelevinsky4}}]
  $F$-polynomials have  a unique monomial of maximal degree. 
  Furthermore, this monomial has coefficient 1, and it is divisible by all the other occurring monomials.
\end{conjecture}
\begin{proof}
  The maximal term is given by the full representaton.
  Note that this is equivalent to the previous conjecture in view of \cite[Proposition 5.3]{fomin-zelevinsky4}.
\end{proof}

\begin{conjecture}[{\cite[Conjecture 6.10]{fomin-zelevinsky4}}]
  About certain integer in the initial-seed mutation formula for $g$-vectors.
\end{conjecture}
\begin{proof}
  Follows from sign coherence of $c$-vectors as in \cite[Proposition 4.2]{NZ12}.
\end{proof}

\begin{conjecture}[{\cite[Conjectures 6.11 and 7.17]{fomin-zelevinsky4}}]
  Formula for $g$-vectors and $d$-vectors in terms of $F$-polynomials.
\end{conjecture}

  \section{TODO list}
  \begin{itemize}
    \item
      Track down the conjectures we will prove:
		
		\begin{itemize}
		  \item CAIV
        \begin{itemize}
          \item C 5.4	F-polynomials have constant term = 1 (follows from sign coherence)
          \item C 5.5	F-polynomial max deg term
          \item C 6.10	initial mutation of g-vectors
          \item C 6.11	g-vectors from F-poly
          \item C 6.13	sign coherence g-vectors
          \item C 7.2	lin ind of cluster monomials
          \item C 7.4	non initial d-vectors are non-negative; d-vector = 0 iff compatible; 	d-vector componentrs depends only on 2 vars
          \item C 7.5	(follows from 7.4)
          \item C 7.6	cluster monomials are parametrized by d-vectors
          \item C 7.10	g-vectors parametrize cluster monomials and form Z-basis
        \end{itemize}
      
      \item RSp framework for cluster algebras
        \begin{itemize}
          \item   C 3.9		seeds containing one cluster variable form a connected graph
          \item   C 3.10		seeds containing given set of c.v. form a connected graph
          \item   C 3.11 3.12	seed determined by c-vectors
          \item   C 3.21		The one we want!
        \end{itemize}
    \end{itemize}


To say:
	d-vectors form a Z-basis (use titling)
	therefore g-vectors form a Z-basis (use RSp 3.21)



    \item
      Write result needed in [RSW]

    \item
      Decide on how much background to give

  \end{itemize}



%  \begin{thebibliography}{99}
%
%  \bibitem[BFZ]{berenstein-fomin-zelevinsky}
%
%  \bibitem[BZ]{berenstein-zelevinsky}
%  A. Berenstein and A. Zelevinsky, ``Quantum cluster algebras.''   {\em Advances in Mathematics}, vol. 195, {\bf 2} (2005), pp.~405--455.
%
%  \bibitem[BMRRT]{buan-marsh-reineke-reiten-todorov}
%  A. Buan, R. Marsh, I. Reiten, M. Reineke, G. Todorov, ``Tilting theory and cluster combinatorics.'' {\em Adv. Math.} \textbf{204} (2006), no. 2, pp.~572–-618.
%
%  \bibitem[CC]{caldero-chapoton}
%  P. Caldero and F. Chapoton, ``Cluster algebras as Hall algebras of quiver representations.'' {\em Comment. Math. Helv.} \textbf{81} (2006), no. 3, pp.~595--616.
%
%  \bibitem[CCS]{caldero-chapoton-schiffler}
%
%  \bibitem[CK]{caldero-keller}
%  P. Caldero and B. Keller, ``From triangulated categories to cluster algebras.''  {\em Invent. Math.} \textbf{172} (2008), no. 1, pp.~169-–211.
%
%  \bibitem[CK2]{caldero-keller2}
%
%  \bibitem[CKLP]{cerulliirelli-keller-labardinifragoso-plamondon}
%
%  \bibitem[D]{demonet}
%
%  \bibitem[DWZ]{derksen-weyman-zelevinsky}
%
%  \bibitem[DR]{dlab-ringel}
%  V. Dlab and C. Ringel, ``Indecomposable Representations of Graphs and Algebras.'' {\em Mem.~Amer. Math. Soc.} \textbf{173} (1976).
%
%  \bibitem[FZ03]{FZ03} S. Fomin and A. Zelevinsky, ``Cluster algebras: Notes for the CDM-03 conference'', in: {\em CDM 2003: Current Developments in Mathematics}, International Press (2004).
%
%  \bibitem[FZ07]{fomin-zelevinsky4} S. Fomin and A. Zelevinsky. ``Cluster algebras. IV. Coefficients.`` {\em Compos. Math.}, \textbf{143}(1):112–164 (2007).
%
%
%  \bibitem[FZ]{fomin-zelevinsky4}
%
%  \bibitem[GLS]{geiss-leclerc-schroer}
%
%  \bibitem[GSV]{gekhtman-shapiro-vainshtein}
%
%  \bibitem[MRZ]{marsh-reineke-zelevinsky}
%
%  \bibitem[NZ12]{NZ12} T. Nakanishi and A. Zelevinsky. ``On tropical dualities in cluster algebras''. In {\em Algebraic groups and quantum groups}, volume 565 of {\em Contemp. Math.}, pages 217–226. Amer. Math. Soc., Providence, RI, (2012).
%
%  \bibitem[Pa]{palu}
%
%  \bibitem[Pl]{plamondon}
%
%  \bibitem[Q]{qin}
%  F. Qin, ``Quantum cluster variables via Serre polynomials.''  {\em J. Reine Angew. Math.} \textbf{668} (2012), pp.~149--190.
%
%  \bibitem[RSp]{reading-speyer}
%  N. Reading and D. Speyer, ``Cambrian fans.'' {\em J. Eur. Math. Soc.} \textbf{11} (2009), no. 2, pp.~407--447.
%
%  \bibitem[RSt]{reading-stella}
%  N. Reading and S. Stella, ``Initial-seed recursions and dualities for $d$-vectors.'' Preprint: arXiv:1409.4723.
%   
%  \bibitem[R1]{rupel1}
%  D. Rupel, ``On a quantum analogue of the Caldero-Chapoton formula.'' {\em Int. Math. Res. Not.} (2011), no. 14, pp.~3207--3236.
%
%  \bibitem[R2]{rupel2}
%  D. Rupel, ``Quantum cluster characters.'' {\em Trans. Amer. Math. Soc.} (2015). DOI: 10.1090/S0002-9947-2015-06251-5.
%
%  \bibitem[W]{wolf}
%  S. Wolf, ``A geometric version of BGP reflection functors''
%
%\end{thebibliography}




\end{document}
