\documentclass{amsart}
\usepackage{amsmath,amssymb,latexsym}

\newtheorem{theorem}{Theorem}
\newtheorem{corollary}[theorem]{Corollary}
\newtheorem{proposition}[theorem]{Proposition}

\newcommand{\bfa}{\mathbf{a}}
\newcommand{\bfb}{\mathbf{b}}
\newcommand{\bfd}{\mathbf{d}}
\newcommand{\bfe}{\mathbf{e}}
\newcommand{\bfg}{\mathbf{g}}
\newcommand{\bfv}{\mathbf{v}}
\newcommand{\bfw}{\mathbf{w}}
\newcommand{\bfX}{\mathbf{X}}
\newcommand{\cA}{\mathcal{A}}
\newcommand{\cF}{\mathcal{F}}
\newcommand{\cQ}{\mathcal{Q}}
\newcommand{\cT}{\mathcal{T}}
\newcommand{\FF}{\mathbb{F}}
\newcommand{\diag}{\operatorname{diag}}
\newcommand{\Ext}{\operatorname{Ext}}
\newcommand{\Gr}{\operatorname{Gr}}
\newcommand{\Hom}{\operatorname{Hom}}
\newcommand{\rep}{\operatorname{rep}}
\newcommand{\ZZ}{\mathbb{Z}}

\title{Some consequences of categorification}
\author[Rupel]{Dylan Rupel}
\address[Dylan Rupel]{
University of Notre Dame\newline
Department of Mathematics\newline
Notre Dame, IN 46556\newline
USA}
\email{drupel@nd.edu}

\begin{document}
\maketitle

 Consider an $n\times n$ skew-symmetrizable matrix $B$ and write $\tilde B=\left[\begin{array}{c}B\\ I_n\end{array}\right]$ for the corresponding principalized exchange matrix.  Fix a choice of diagonal symmetrizing matrix $D=\diag(d_1,\ldots,d_n)$ with $d_i\in\ZZ_{>0}$, i.e. such that $DB$ is skew-symmetric.  Let $\Lambda$ be a skew-symmetric $2n\times2n$ matrix compatible with $\tilde B$, e.g. we may take $\Lambda=\left[\begin{array}{cc}\boldsymbol{0} & -D\\ D & -DB\end{array}\right]$ so that $\tilde B^T\Lambda=\big[\,D\ \boldsymbol{0}\,\big]$.  

 Let $v$ be a formal variable.  We will work inside the $2n$-dimensional quantum torus algebra $\cT:=\cT_{\Lambda,v}$ with a $\ZZ[v^{\pm1}]$-basis consisting of vectors $X^\bfa$ for $\bfa\in\ZZ^{2n}$ and multiplication given by $X^\bfa X^{\bfa'}=v^{\Lambda(\bfa,\bfa')}X^{\bfa+\bfa'}$ for $\bfa,\bfa'\in\ZZ^{2n}$, where we abuse notation slightly and write $\Lambda:\ZZ^{2n}\times\ZZ^{2n}\to\ZZ$ for the associated bilinear form.  The compatibility condition guarantees that $X^{\bfb^k}$ and $X_\ell:=X^{\varepsilon_\ell}$ commute for $k\ne\ell$, where $\bfb^k$ denotes the $k^{th}$ column of $\tilde B$ and $\varepsilon_\ell$ denotes the standard basis vector of $\ZZ^{2n}$.  Write $\bfb^k=\bfb^k_+-\bfb^k_-$ for minimal positive vectors $\bfb^k_+,\bfb^k_-\in\ZZ^{2n}$.

 The pair $\Sigma=(\bfX,\tilde B)$ where $\bfX=(X_1,\ldots,X_{2n})$ is called a \emph{quantum seed}.  For each $1\le k\le n$, we may define a new quantum seed $\mu_k\Sigma=(\bfX',\tilde B')$ by \emph{mutation in direction $k$}, where $X'_i=X_i$ for $i\ne k$ while $X'_k$ is given by the \emph{exchange relation}:
 \[X'_k=X^{\bfb_+^k-\varepsilon_k}+X^{\bfb_-^k-\varepsilon_k}\]
 and $\tilde B'=(b'_{ij})$ is given by
 \[b'_{ij}=\begin{cases}-b_{ij} & \text{if $i=k$ or $j=k$;}\\ b_{ij}+b_{ik}[b_{kj}]_++[-b_{ik}]_+b_{kj} & \text{otherwise.}\end{cases}\]
 Call quantum seeds $\Sigma$ and $\Sigma'$ \emph{mutation equivalent} if one can be obtained from the other by a sequence of mutations.

 Since $\cT$ is an Ore domain it has a well-defined skew-field of fractions $\cF$.  Clearly the exchange relations imply that each quantum cluster variable $X'_i$ is contained in $\cF$ where $X'_i$ belongs to a quantum seed $\Sigma'=(\bfX',\tilde B')$ mutation equivalent to $\Sigma$.  The \emph{quantum cluster algebra} $\cA(\bfX,\tilde B)\subset\cF$ is the $\ZZ[v^{\pm1}]$-subalgebra generated by all quantum cluster variables belonging to quantum seeds mutation equivalent to $\Sigma=(\bfX,\tilde B)$.

 The first important structural result on quantum cluster algebras is the \emph{Quantum Laurent Phenomenon}.
 \begin{theorem}\cite{berenstein-zelevinsky}
  $\cA(\bfX,\tilde B)\subset\cT$
 \end{theorem}
 This raises the natural question: how can one describe the Laurent expansions of the cluster variables.  To answer this question we assume from now on that $B$ is \emph{acyclic}, i.e. if $i<j$ then $b_{ij}<0$.  In other words the associated quiver $Q:=Q_{\tilde B}$ with vertices $\{1,\ldots,n\}$ and $\gcd(b_{ij},b_{ji})$ arrows $i\longrightarrow j$ whenever $b_{ij}<0$ has no oriented cycles.  We will need the principalized quiver $\tilde Q$ obtained from $Q$ by attaching to each vertex $i$ an additional vertex $n+i$ via a single arrow $i\to n+i$.  Then the Laurent expansions of non-initial quantum cluster variables can be described in terms of the representation theory of the valued quiver $(\tilde Q,\tilde D)$ where $\tilde D=\left[\begin{array}{cc}D&0\\0&D\end{array}\right]$.

 Write $\FF_q$ for the finite field with $q$ elements and specialize $v=\sqrt{q}$.  Choose an algebraic closure $\overline{\FF}_q$ and write $\FF_{q^d}$ for the degree $d$ extension of $\FF_q$ inside $\overline{\FF}_q$.  Note that this provides a canonical identification of $\FF_{q^d}$ as an $\FF_{q^g}$-vector space whenever $g|d$.  A representation $V=(V_i,V_a)$ of $(Q,D)$ consists of an $\FF_{q^{d_i}}$-vector space $V_i$ for each vertex $i$ of $Q$ and an $\FF_{q^{\gcd{d_i,d_j}}}$-linear map $V_a:V_i\to V_j$ for each arrow $a:i\to j$ (i.e. when $b_{ij}<0$).  The finite-dimensional representations of $(Q,D)$ form a hereditary abelian category denoted by $\rep_{\FF_q}\!(Q,D)$.  The category $\rep_{\FF_q}\!(Q,D)$ naturally embeds into the category $\rep_{\FF_d}\!(\tilde Q,\tilde D)$, it is this subcategory that we will be interested in and thus always consider $\rep_{\FF_q}\!(Q,D)$ to be this subcategory of $\rep_{\FF_d}\!(\tilde Q,\tilde D)$.  

 Write $\tilde\cQ$ for the Grothendieck group of $\rep_{\FF_q}\!(\tilde Q,\tilde D)$ and let $\cQ\subset\tilde\cQ$ denote the Grothendieck group of $\rep_{\FF_q}\!(Q,D)$.  Then since $\tilde Q$ is acyclic, the Euler-Ringel form $\langle V,W\rangle:=\dim_{\FF_q}\Hom(V,W)-\dim_{\FF_q}\Ext(V,W)$ only depends on the classes of $V$ and $W$ in $\tilde\cQ$ and thus may be given on the classes $\alpha_i=[S_i]$ of the vertex-simple representations by
 \[\langle\alpha_i,\alpha_j\rangle=\begin{cases} d_i & \text{if $i=j$;}\\d_ib_{ij} & \text{if $b_{ij}<0$;}\\-d_i & \text{if $j=n+i$;}\\0 & \text{otherwise.}\end{cases}\]
 For $\bfe\in\cQ$ define ${}^*\bfe=\sum\limits_{i=1}^{2n}\frac{1}{d_i}\langle\alpha_i,\bfe\rangle\cdot\alpha_i$ and $\bfe^*=\sum\limits_{j=1}^{2n}\frac{1}{d_j}\langle\bfe,\alpha_j\rangle\cdot\alpha_j$.  Then the \emph{quantum cluster character} of a representation $V\in\rep_{\FF_q}(Q,D)$ with $[V]=:\bfv$ is given by
 \[X_V=\sum\limits_{\bfe\in\cQ} v^{\langle\bfe,\bfv-\bfe\rangle}\big|\Gr_\bfe(V)\big|X^{\tilde B\bfe-{}^*\bfv}\in\cT\]
 where $\Gr_\bfe(V)$ denotes the set of all submodules $E\subset V$ with isomorphism class $[E]=\bfe$.  Note that $\tilde B\bfe={}^*\bfe-\bfe^*$ and for $1\le j\le n$ we have ${}^*\alpha_j=\alpha_j+\sum\limits_{i:b_{ij}<0}b_{ij}\alpha_i$.

 A representation $V$ is called \emph{rigid} if $\Ext(V,V)=0$.
 \begin{theorem}\cite{qin,rupel2}
  For each rigid indecomposable $V\in\rep_{\FF_q}(Q,D)$, the quantum cluster character $X_V$ is a quantum cluster variable of $\cA(\bfX,\tilde B)$.  Moreover, every non-initial cluster variable of $\cA(\bfX,\tilde B)$ arises in this way.
 \end{theorem}

 The connection between the representation theory of $(Q,D)$ and the cluster algebra is actually much stronger.  For a source (resp. sink) vertex $k$ in $Q$, write $\Sigma_k^-:\rep_{\FF_q}(Q,D)\to\rep_{\FF_q}(\mu_kQ,D)$ (resp. $\Sigma_k^+:\rep_{\FF_q}(Q,D)\to\rep_{\FF_q}(\mu_kQ,D)$) for the reflection functor as defined in \cite[Sec. 2]{dlab-ringel}.  We will drop the adornment and simply write $\Sigma_k$ for both reflection functors, which functor to apply will be clear from context.
 \begin{theorem}\cite{rupel1}\label{th:reflection functor}
  Let $k$ be a sink or a source in $Q$ and write $(\bfX',\tilde B')=\mu_k(\bfX,\tilde B)$.  For any representation $V\in\rep_{\FF_q}(Q,D)$ we have $X_V=X'_{\Sigma_kV}$ inside $\cF$, where $X'_{\Sigma_kV}$ denotes the quantum cluster character of $\Sigma_kV\in\rep_{\FF_q}(\mu_kQ,D)$ in the variables $\bfX'$.
 \end{theorem}

 \begin{proposition}
  The denominator vector of $X_V$ is precisely the dimension vector $\bfv$ of $V$.
 \end{proposition}
 \begin{proof}
  The proof is identical to that of \cite[Sec. 4, Cor. 2]{caldero-keller} with appropriate modifications in the valued quiver setting, we recall the details here for convenience of the reader.  

  First note that for any representation $W$ with $[W]=\bfw$ we have $\frac{1}{d_i}\langle W,I_i\rangle=w_i=\frac{1}{d_i}\langle P_i,W\rangle$ where $I_i$ and $P_i$ denote respectively the injective hull and projective cover of the vertex simple $S_i$.  Next using the injective resolution of $S_i$ we see that $\langle W,S_i\rangle\le\langle W,I_i\rangle$ while using the projective resolution gives $\langle S_i,W\rangle\le\langle P_i,W\rangle$.  Now consider a subrepresentation $E\subset V$ with $[E]=\bfe$, applying the above considerations we see that the $i^{th}$ component of $\bfe^*+{}^*(\bfv-\bfe)$ is bounded by the $i^{th}$ component of $\bfv$:
  \[\frac{1}{d_i}\langle E,S_i\rangle+\frac{1}{d_i}\langle S_i,V/E\rangle\le\frac{1}{d_i}\langle E,I_i\rangle+\frac{1}{d_i}\langle P_i,V/E\rangle=e_i+(v_i-e_i)=v_i.\]

  To finish the proof, for each $1\le i\le n$ we must exhibit a subrepresentation $E\subset V$ which realizes this bound.  To construct such a subrepresentation, let $J_{i\to j}$ be the set of all vertices $j$ in $Q$ for which there exists a path (possibly trivial) beginning at $i$ and ending at $j$.  Now set $E_j=V_j$ for $j\in J_{i\to j}$ and $E_j=0$ for $j\notin J_{i\to j}$.  Recall that in an injective resolution $0\longrightarrow S_i\longrightarrow I_i\longrightarrow I\longrightarrow 0$ the injective representation $I$ has nonzero components only at vertices $j$ which admit a nontrivial path \emph{to} vertex $i$, that is while in a projective resolution $0\longrightarrow P\longrightarrow P_i\longrightarrow S_i\longrightarrow 0$ the projective representation $P$ has nonzero components only at vertices $j$ which admit a nontrivial path \emph{from} vertex $i$.  It follows that $\langle E,S_i\rangle=\langle E,I_i\rangle-\langle E,I\rangle=\langle E,I_i\rangle$ and $\langle S_i,V/E\rangle=\langle P_i,V/E\rangle-\langle P,V/E\rangle=\langle P_i,V/E\rangle$ and thus 
  \[\frac{1}{d_i}\langle E,S_i\rangle+\frac{1}{d_i}\langle S_i,V/E\rangle=\frac{1}{d_i}\langle E,I_i\rangle+\frac{1}{d_i}\langle P_i,V/E\rangle=e_i+(v_i-e_i)=v_i\]
  as desired.
 \end{proof}

 \begin{corollary}
  The mutation of the initial cluster at a sink or a source vertex transforms denomiator vectors according to the simple reflection associated to that vertex.
 \end{corollary}
 \begin{proof}
  By Theorem~\ref{th:reflection functor}, the initial cluster mutation at a sink or a source vertex transforms quantum cluster characters according to the associated reflection functor on $\rep_{\FF_q}(Q,D)$.  The result then follows from \cite[Prop. 2.1]{dlab-ringel}.
 \end{proof}

 Let $E=(e_{ij})$ be the matrix given by
 \[e_{ij}=\begin{cases} -1 & \text{if $i=j$;}\\ [-b_{ij}]_+ & \text{if $i\ne j$.}\end{cases}\]
 \begin{theorem}
  Let $\bfd_i^t\in\ZZ^n$ be a denominator vector for a cluster variable of $\cA(\bfX,\tilde B)$ and write $\bfg_i^t$ for the $g$-vector of that cluster variable.  Then $\bfg_i^t=E\bfd_i^t$.
 \end{theorem}
 \begin{proof}
  Notice that the $g$-vector of the cluster variable $X_V$ is given by $-{}^*\bfv$ where $\bfv=[V]$.  The result then immediately follows from the definitions of $E$ and the ${}^*(-)$ operator.
 \end{proof}

 \section{Possible other directions}
 \begin{itemize}
   \item Can I use categorification to prove any part of Conj. 7.4 from [FZ4]?
 \end{itemize}


\begin{thebibliography}{99}

\bibitem[BZ]{berenstein-zelevinsky}
A. Berenstein and A. Zelevinsky, ``Quantum cluster algebras.''   {\em Advances in Mathematics}, vol. 195, {\bf 2} (2005), pp.~405--455.

 \bibitem[BMRRT]{bmrrt}
 A. Buan, R. Marsh, I. Reiten, M. Reineke, G. Todorov, ``Tilting theory and cluster combinatorics.'' {\em Adv. Math.} \textbf{204} (2006), no. 2, pp.~572–-618.

 \bibitem[CC]{caldero-chapoton}
 P. Caldero and F. Chapoton, ``Cluster algebras as Hall algebras of quiver representations.'' {\em Comment. Math. Helv.} \textbf{81} (2006), no. 3, pp.~595--616.

 \bibitem[CK]{caldero-keller}
 P. Caldero and B. Keller, ``From triangulated categories to cluster algebras.''  {\em Invent. Math.} \textbf{172} (2008), no. 1, pp.~169-–211.

 \bibitem[DR]{dlab-ringel}
 V. Dlab and C. Ringel, ``Indecomposable Representations of Graphs and Algebras.'' {\em Mem.~Amer. Math. Soc.} \textbf{173} (1976).

 \bibitem[Q]{qin}
 F. Qin, ``Quantum cluster variables via Serre polynomials.''  {\em J. Reine Angew. Math.} \textbf{668} (2012), pp.~149--190.

 \bibitem[RSp]{reading-speyer}
 N. Reading and D. Speyer, ``Cambrian fans.'' {\em J. Eur. Math. Soc.} \textbf{11} (2009), no. 2, pp.~407--447.

 \bibitem[RSt]{reading-stella}
 N. Reading and S. Stella, ``Initial-seed recursions and dualities for $d$-vectors.'' Preprint: arXiv:1409.4723.
   
 \bibitem[R1]{rupel1}
 D. Rupel, ``On a quantum analogue of the Caldero-Chapoton formula.'' {\em Int. Math. Res. Not.} (2011), no. 14, pp.~3207--3236.

 \bibitem[R2]{rupel2}
 D. Rupel, ``Quantum cluster characters.'' {\em Trans. Amer. Math. Soc.} (2015). DOI: 10.1090/S0002-9947-2015-06251-5.

\end{thebibliography}
\end{document}