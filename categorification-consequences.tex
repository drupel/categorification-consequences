\documentclass[12pt]{amsart}
% this is here to force arXiv to produce a nice output
\pdfoutput=1

\usepackage{mathtools}
\usepackage{amsmath}
\usepackage{amsthm}
\usepackage{amssymb}
\usepackage{amsbsy}
\usepackage{amstext}
\usepackage{amsopn}
\usepackage{enumerate}
\usepackage{xcolor}
\usepackage{graphicx}
\usepackage{microtype}
\usepackage{verbatim}
\usepackage[margin=1in,marginparwidth=0.8in, marginparsep=0.1in]{geometry}
\renewcommand{\baselinestretch}{1.2} % changes page formatting
\usepackage[pagebackref, bookmarks=true, bookmarksopen=true, bookmarksdepth=3,bookmarksopenlevel=2, colorlinks=true, linkcolor=blue, citecolor=blue, filecolor=blue, menucolor=blue, urlcolor=blue]{hyperref}
\usepackage{newtxtext} % changes font appearance, replaces times
\usepackage{stmaryrd}
\usepackage{accents}
\usepackage{bbm}
\usepackage{tikz}

\input xy
\xyoption{all}

% Commands for marginal notes below
\usepackage[draft]{say}
\newcommand{\sayD}[1]{\say[D]{#1}}
\newcommand{\sayS}[1]{\say[S]{#1}}

\newtheorem{theorem}{Theorem}
\newtheorem{corollary}[theorem]{Corollary}
\newtheorem{conjecture}[theorem]{Conjecture}
\newtheorem{proposition}[theorem]{Proposition}
\newtheorem{remark}[theorem]{Remark}

\newcommand{\bfa}{\mathbf{a}}
\newcommand{\bfb}{\mathbf{b}}
\newcommand{\bfc}{\mathbf{c}}
\newcommand{\bfd}{\mathbf{d}}
\newcommand{\bfe}{\mathbf{e}}
\newcommand{\bfg}{\mathbf{g}}
\newcommand{\bfv}{\mathbf{v}}
\newcommand{\bfw}{\mathbf{w}}
\newcommand{\bfx}{\mathbf{x}}
\newcommand{\bfX}{\mathbf{X}}
\newcommand{\cA}{\mathcal{A}}
\newcommand{\cF}{\mathcal{F}}
\newcommand{\cQ}{\mathcal{Q}}
\newcommand{\cT}{\mathcal{T}}
\newcommand{\FF}{\mathbb{F}}
\newcommand{\diag}{\operatorname{diag}}
\newcommand{\Ext}{\operatorname{Ext}}
\newcommand{\Gr}{\operatorname{Gr}}
\newcommand{\half}{{\frac{1}{2}}}
\newcommand{\Hom}{\operatorname{Hom}}
\newcommand{\into}{\hookrightarrow}
\newcommand{\onto}{\twoheadrightarrow}
\newcommand{\rep}{\operatorname{rep}}
\newcommand{\ZZ}{\mathbb{Z}}
\newcommand{\newword}[1]{\textbf{\emph{#1}}}
%\renewcommand{\thesubsection}{\arabic{subsection}}
%\makeatletter
%\def\@seccntformat#1{\@ifundefined{#1@cntformat}%
%   {\csname the#1\endcsname\quad}%       default
%   {\csname #1@cntformat\endcsname}}%    enable individual control
%\newcommand\section@cntformat{}
%\makeatother

\newenvironment{enumeratea}{\begin{enumerate}[\upshape (a)]}{\end{enumerate}}

\title{Some consequences of categorification}
\author[Rupel]{Dylan Rupel}
\address[Dylan Rupel]{ University of Notre Dame, Department of Mathematics, Notre Dame, IN 46556, USA}
\email{drupel@nd.edu}

\author[Stella]{Salvatore Stella}
\address[Salvatore Stella]{University of Haifa, Departments of Mathematics, Haifa, Mount Carmel, 31905, Israel}
\email{stella@math.haifa.ac.il}

\begin{document}
\begin{abstract}
  Several conjectures on acyclic skew-symmetrizable cluster algebras are proven as direct consequences of their categorification via valued quivers.
  These include conjectures of Fomin-Zelevinsky, Reading-Speyer, and Reading-Stella related to $\bfd$-vectors, $\bfg$-vectors, and $F$-polynomials.
\end{abstract}
\maketitle

%%%%%%%%%%%%%%%%%%%%%%
\section{Introduction}
  The categorification of skew-symmetric cluster algebras using representations of quivers was initiated by Marsh, Reineke, and Zelevinsky \cite{marsh-reineke-zelevinsky} for Dynkin quivers.  
  With the advent of cluster characters \cite{caldero-chapoton}, the subject has exploded as an industry all its own, leading to the publication of numerous articles including \cite{caldero-chapoton-schiffler,buan-marsh-reineke-reiten-todorov,derksen-weyman-zelevinsky,geiss-leclerc-schroer,caldero-keller,caldero-keller2,plamondon,palu,rupel1,qin,rupel2} just to name a few.  

  The main idea is to understand the combinatorics of cluster mutations in terms of a category of representations of a quiver, and in particular to obtain an interpretation of Laurent expansions of non-initial cluster variables as cluster characters.
 These are generating functions for certain geometric invariants (e.g. Euler characteristics, point counts over finite fields, Poincar\'e polynomials, etc.) of varieties of subrepresentations in an associated quiver representation.
  
  Using categorification, many structural conjectures on skew-symmetric cluster algebras, their cluster variables and the associated combinatorics of mutations hae been established.
  The main goal of this note is to observe that the categorification of acyclic skew-symmetrizable (quantum) cluster algebras \cite{rupel1,rupel2} naturally leads to proofs of many of the same conjectures.
  \sayS{commented out ``as well as providing proofs of some conjectures not obtainable from the standard quiver (with potential) approach'' because I do not understand it.}  

  Alternative approaches to categorification of skew-symmetrizable cluster algebras using quivers with automorphism or group species with potential have been introduced by Demonet \cite{demonet1,demonet2}. 
  \sayS{Shall we cite Daniel?}
  These formalisms have allowed him to prove many conjectures on cluster algebras that we thus omit here even though they should be deducible in our setting.

  Since the appearance of Demonet's works, new conjectures on cluster algebras, particularly ones related to denominator vectors, have been made and our goal here is to resolve several of these.
  In some cases our proofs will follow by direct translation of arguments given in the skew-symmetric case. 
  \sayS{most instead of some?}
  \sayS{Commented out old text}
  % The proofs of some results are direct translations of proofs from the skew-symmetric case, while others lie entirely in the realm of the valued quiver approach.

%%%%%%%%%%%%%%%%%%%%%%%%
\textsc{Acknowledgments}
  We are grateful to Giovanni Cerulli-Irelli for some insight on references.
  D.R. was partially supported by an AMS-Simons Travel Grant; S.S. was supported by ISF grant 1144/16.

\section{Quantum Cluster Algebras and Quantum Cluster Characters}
  
  In the interest of brevity we will assume that the reader is familiar with the basic notions in the theory of cluster algebras and refer to \cite{fomin-zelevinsky4} for background material.

  Let $B$ be a $n\times n$ skew-symmetrizable integer matrix; this means that there exists a diagonal matrix $D$ with positive integers on its diagonal such that $DB$ is skew-symmetric.
  We assume thorough the paper that $B$ is \newword{acyclic} i.e. up to a simultaneous permutation of rows and columns $B$ has only non-negative entries above the diagonal.
  Let $\widetilde{B}$ be the matrix obtained by stacking a $n\times n$ identity matrix below $B$ and write $\cA(\bfx,\widetilde{B})$ for the associated cluster algebra; it is a cluster algebra with \newword{principal coefficients}.

  Since $\widetilde{B}$ has full rank, the algebra $\cA(\bfx,\widetilde{B})$ admits a log-canonical Poisson structure compatible with mutations \cite{gekhtman-shapiro-vainshtein} and one may obtain a quantum cluster algebra via a remarkably simple deformation quantization \cite{berenstein-zelevinsky}.  

  Intuitively, there are only few points to keep in mind.
  Fix an indetermiate $q$, then the definition of the quantum cluster algebra $\cA_q(\bfX,\widetilde B)$ can be obtained from that of $\cA(\bfx,\widetilde B)$ via the following modifications:
  \begin{itemize}
    \item 
      The initial cluster $\bfX$ consists of $2n$ variables which $q$-commute, i.e. there exists a skew-symmetric matrix $\Lambda=(\lambda_{ij})$ so that $X_iX_j=q^{\lambda_{ij}}X_jX_i$.  
    
    \item 
      Let $\cT_\Lambda$ be the quantum torus algebra generated by $X_i^{\pm1}$, $1\le i\le 2n$, over the ring $\ZZ[q^{\pm\half}]$.
      It admits an anti-involution (called the \newword{bar-involution}) fixing each $X_i$ and interchanging $q$ with $q^{-1}$.
      Every non-initial cluster variable should also be ``bar-invariant'' and this uniquely determines the power of $q^\half$ by which to multiply each monomial in the exchange relations.  
      More precisely, using the bar-invariant monomial basis $X^\bfa$, $\bfa\in\ZZ^{2n}$, of $\cT_\Lambda$ given by
      \[
        X^\bfa=q^{-\half\sum\limits_{i<j}\lambda_{ij}a_ia_j}X_1^{a_1}\cdots X_{2n}^{a_{2n}},
      \]
      we may write the mutated variables as $X'_k=X^{\bfb_+^k-\varepsilon_k}+X^{\bfb_-^k-\varepsilon_k}$, where $\varepsilon_k$ denotes the standard basis vector of $\ZZ^{2n}$ and the $k^{th}$ column of $\widetilde{B}$ decomposes as $\bfb^k=\bfb^k_+-\bfb^k_-$ for minimal non-negative vectors $\bfb^k_+,\bfb^k_-\in\ZZ_{\ge0}^{2n}$.

    \item 
      Each cluster $\bfX'$ obtainable from $\bfX$ by a sequence of mutations should again generate a quantum torus, i.e. consist of $q$-commuting variables.
      This forces the compatibility condition $\tilde B^T\Lambda=\big[\,D\ \boldsymbol{0}\,\big]$ on the commutation matrix $\Lambda$ and the exchange matrix $\tilde B$.
    An easy calculation shows that this compatibility naturally reproduces under mutations.


    Let $\Lambda$ be a skew-symmetric $2n\times2n$ matrix compatible with $\tilde B$, e.g. given any $n\times n$ skew-symmetric matrix $\Lambda_0$ we may take $\Lambda=\left[\begin{array}{cc}\Lambda_0 & -\Lambda_0B-D\\ -B^T\Lambda_0+D & B^T\Lambda_0B+B^TD\end{array}\right]$ so that $\tilde B^T\Lambda=\big[\,D\ \boldsymbol{0}\,\big]$ (this is the required compatibility condition).

  \end{itemize}
  Note that the final compatibility condition is identical to the condition required of a compatible Poisson structure.
  
  Assume the pair $(\tilde B,\Lambda)$ is compatible.
  Let $\cF_\Lambda$ denote the skew-field of fractions of $\cT_\Lambda$ (cf. \cite{berenstein-zelevinsky}).


  By work of Berenstein and Zelevinsky \cite{berenstein-zelevinsky}, the famous Laurent phenomenon holds in this adapted setting and one may ask how to describe the initial cluster Laurent expansions of non-initial quantum cluster variables.  
  The answer will be given using the representation theory of valued quivers in the following section.

  By work of Tran \cite{tran} which extends the theory of $F$-polynomials from \cite{fomin-zelevinsky4} to the quantum case, it suffices to consider the principal coefficients cluster algebras defined as follows.  
  \sayD{Add something about $F$-polynomials and $g$-vectors here.}

  \section{Quantum Cluster Characters}
  Define a quiver $Q_B$ with vertices $\{1,\ldots,n\}$ and $\gcd(b_{ij},b_{ji})$ arrows $i\longrightarrow j$ whenever $b_{ij}<0$, we assume this quiver to be acyclic.
  Write $\tilde Q$ for the quiver obtained from $Q_B$ by attaching to each vertex $i$ an additional vertex $n+i$ via a single arrow $i\to n+i$.  
  The Laurent expansions of non-initial quantum cluster variables can be described in terms of the representation theory of the valued quiver $(\tilde Q,\tilde D)$ with $\tilde D=\left[\begin{array}{cc}D&0\\0&D\end{array}\right]$.

  Let $\FF$ be a finite field and fix an algebraic closure $\overline{\FF}$ of $\FF$.  
  Write $\FF^{\langle d\rangle}$ for the degree $d$ extension of $\FF$ inside $\overline{\FF}$.
  Note that this provides a canonical identification of $\FF^{\langle d\rangle}$ as a vector space over $\FF^{\langle g\rangle}$ whenever $g|d$.  
  A representation $V=(V_i,V_a)$ of $(Q,D)$ consists of an $\FF^{\langle d_i\rangle}$-vector space $V_i$ for each vertex $i$ and an $\FF^{\langle\gcd(d_i,d_j)\rangle}$-linear map $V_a:V_i\to V_j$ for each arrow $a:i\to j$ (i.e. when $b_{ij}<0$).  
  The finite-dimensional representations of $(Q,D)$ form a hereditary abelian category denoted by $\rep_\FF(Q,D)$.  
  The category $\rep_\FF(Q,D)$ naturally embeds into the category $\rep_\FF(\tilde Q,\tilde D)$, it is this subcategory that we will be interested in and thus always consider $\rep_\FF(Q,D)$ to be this full subcategory of $\rep_\FF(\tilde Q,\tilde D)$.  


  Write $\tilde\cQ$ for the Grothendieck group of $\rep_\FF(\tilde Q,\tilde D)$ and let $\cQ\subset\tilde\cQ$ denote the Grothendieck group of $\rep_\FF(Q,D)$.
  Since $\tilde Q$ is acyclic, there is an identification $\tilde\cQ\cong\ZZ^{2n}$ by taking classes $\alpha_i=[S_i]$ of the vertex-simple representations as a basis.
  The Euler-Ringel form $\langle V,W\rangle:=\dim_\FF\Hom(V,W)-\dim_\FF\Ext(V,W)$ only depends on the classes of $V$ and $W$ in $\tilde\cQ$.  
  More precisely, the Euler-Ringel form may be computed in the basis of vertex-simples by
  \[\langle\alpha_i,\alpha_j\rangle=\begin{cases} d_i & \text{if $i=j$;}\\d_ib_{ij} & \text{if $b_{ij}<0$;}\\-d_i & \text{if $j=n+i$;}\\0 & \text{otherwise.}\end{cases}\]
  For $\bfe\in\cQ$, define ${}^*\bfe=\sum\limits_{i=1}^n\frac{1}{d_i}\langle\alpha_i,\bfe\rangle\cdot\alpha_i\in\cQ$ and $\bfe^*=\sum\limits_{j=1}^{2n}\frac{1}{d_j}\langle\bfe,\alpha_j\rangle\cdot\alpha_j\in\tilde\cQ$, noting that these do not depend on the choice of symmetrizing matrix $D$. 
  Then the \emph{quantum cluster character} of a representation $V\in\rep_\FF(Q,D)$ with $[V]=:\bfv$ is given by \[X_V=\sum\limits_{\bfe\in\cQ} |\FF|^{-\half\langle\bfe,\bfv-\bfe\rangle}\big|\!\Gr_\bfe(V)\big|X^{\tilde B\bfe-{}^*\bfv}\in\cT_\Lambda,\] where $\Gr_\bfe(V)$ denotes the set of all subrepresentations $E\subset V$ with isomorphism class $[E]=\bfe$.  
  Note that $\tilde B\bfe={}^*\bfe-\bfe^*$ and, for $1\le j\le n$, we have ${}^*\alpha_j=\alpha_j+\sum\limits_{i:b_{ij}<0}b_{ij}\alpha_i$.

  A representation $V$ is called \emph{rigid} if $\Ext(V,V)=0$.  
  \begin{theorem}\cite{rupel1,rupel2}
    \label{th:quantum cluster characters}\mbox{}
    \sayD{Upgrade to include cluster monomials.}
    \begin{enumeratea}
      \item For each rigid representation $V\in\rep_\FF(Q,D)$, the quantum cluster character $X_V$ is a quantum cluster monomial of $\cA_{|\FF|}(\bfX,\tilde B)$.  
      Moreover, every non-initial cluster variable of $\cA_{|\FF|}(\bfX,\tilde B)$ arises in this way from a rigid indecomposable representation.
      \item For each $\bfe\in\cQ$ and each positive root $\bfv\in\cQ$, there exists a polynomial $P_{\bfv,\bfe}(q)$ such that for any rigid indecomposable $V\in\rep_\FF(Q,D)$ with isomorphism class $\bfv$, we have $\big|Gr_\bfe(V)\big|=P_{\bfv,\bfe}\big(|\FF|\big)$.  These polynomials give a ``generic'' quantum cluster character 
      \[X_\bfv=\sum\limits_{\bfe\in\cQ} q^{-\half\langle\bfe,\bfv-\bfe\rangle}P_{\bfv,\bfe}(q)X^{\tilde B\bfe-{}^*\bfv}\]
      which computes the cluster variables of $\cA_q(\bfX,\tilde B)$ with arbitrary parameter $q$.
    \end{enumeratea}
  \end{theorem}
  An analogous result was obtained for acyclic skew-symmetric quantum cluster algebras by Qin \cite{qin} using representations of acyclic quivers and with counting polynomials replaced by Poincar\'e polynomials.  Setting $q=1$ in Theorem~\ref{th:quantum cluster characters}(b) gives the following result from which we will deduce the main results of this note.
  \begin{corollary}
    \label{cor:classical cluster characters}
    All non-initial cluster variables of the cluster algebra $\cA(\tilde B)$ are computed by the cluster characters
    \[x_\bfv=x^{-{}^*\bfv}\sum\limits_{\bfe\in\cQ} P_{\bfv,\bfe}(1)x^{\tilde B\bfe}\]
    as $\bfv$ ranges over all positive roots in the root system of the Cartan companion of $B$.
    In particular, the $g$-vector is given by $-{}^*\bfv$ and the $F$-polynomial is $\sum\limits_{\bfe\in\cQ} P_{\bfv,\bfe}(1)y^\bfe$.
    %Moreover, $F$-polynomials have constant term 1 and all coefficients are positive.
  \end{corollary}
  {\color{lightgray}
  \begin{remark}
    It immediately follows that the $F$-polynomials have constant term 1 since $P_{\bfv,0}\equiv1$ for any $\bfv\in\cQ$.
    This proves \cite[Conj. 5.4]{fomin-zelevinsky4}.
    It immediately follows also that \cite[Conj. 5.5]{fomin-zelevinsky4} holds for acyclic initial seeds.
  \end{remark}
}
  \begin{proposition}
    \label{prop:g index}
    The $\bfg$-vector $-{}^*\bfv$ of the cluster variable $x_\bfv$ is precisely the index of the rigid representation $V$ with dimension vector $\bfv$.
  \end{proposition}
  {\color{lightgray}
  \begin{theorem}
    Let $V$ and $V'$ be rigid representations of $(Q,D)$.
    If $V\oplus V'$ is rigid, then the $\bfg$-vectors $-{}^*\bfv$ and $-{}^*\bfv'$ are sign-coherent.
  \end{theorem}
  \begin{proof}
    We will show that $V\oplus V'$ cannot be rigid if the $\bfg$-vectors of $V$ and $V'$ are not sign-coherent.
    Consider minimial injective copresentations
    \[\xymatrix{0 \ar[r] & V \ar[r]^{\iota_0} & J_0 \ar[r] & J_1 \ar[r] & 0}\]
    and
    \[\xymatrix{0 \ar[r] & V' \ar[r] & J'_0 \ar[r] & J'_1 \ar[r] & 0}.\]
    Assume that there exists an injective $I$ which is a summand of both $J_0$ and $J'_1$, i.e.\ by Proposition~\ref{prop:g index} we are assuming the $\bfg$-vectors are not sign-coherent.
    Write $\iota_1:I\into J'_1$ and $\pi_0:J_0\onto I$ for the corresponding split inclusion and projection maps.
    Since the injective copresentation of $V'$ is minimal, the upper pullback sequence below must be non-split:
    \begin{equation}
      \label{dia:I pullback}
      \xymatrix{0\ar[r]& V'\ar[r]\ar@{=}[d]& F\ar[d]\ar[r]\ar@{}[dr]|<{\lrcorner}& I\ar[d]^{\iota_1}\ar[r]& 0\\ 0\ar[r]& V'\ar[r]& J'_0\ar[r]& J'_1\ar[r]& 0.}
    \end{equation}
    Taking pullbacks again and recalling that $\pi_0$ is a split epimorphism, we see that the upper sequence below is also non-split:
    \begin{equation}
      \label{dia:J_0 pullback}
      \xymatrix{0\ar[r]& V'\ar[r]\ar@{=}[d]& E\ar[d]\ar[r]\ar@{}[dr]|<{\lrcorner}& J_0\ar[d]^{\pi_0}\ar[r]& 0\\ 0\ar[r]& V'\ar[r]& F\ar[r]& I\ar[r]& 0.}
    \end{equation}
    By minimality of the injective copresentation, the map $\iota_0:V\into J_0$ is left minimal and $J_0$ is the maximal essential extension of $V$.

    Claim: These imply the upper pullback sequence below is not split:
    \begin{equation}
      \label{dia:V pullback}
      \xymatrix{ & & 0\ar[d] & 0\ar[d] &\\ 0\ar[r]& V'\ar[r]\ar@{=}[d]& \tilde E\ar[d]\ar[r]\ar@{}[dr]|<{\lrcorner}& V\ar[d]^{\iota_0}\ar[r]& 0\\ 0\ar[r]& V'\ar[r]& E\ar[r]\ar[d]& J_0\ar[r]\ar[d]& 0\\ & & J_1\ar[d]\ar@{=}[r] & J_1\ar[d] &\\ & & 0 & 0 &.}
    \end{equation}
  \end{proof}
}

  The connection between the representation theory of $(Q,D)$ and the cluster algebra is actually much stronger.  For a source (resp. sink) vertex $k$ in $Q$, write $\Sigma_k^-:\rep_\FF(Q,D)\to\rep_\FF(\mu_kQ,D)$ (resp. $\Sigma_k^+:\rep_\FF(Q,D)\to\rep_\FF(\mu_kQ,D)$) for the reflection functor as defined in \cite[Sec. 2]{dlab-ringel}.  
  In what follows, we will drop the adornment and simply write $\Sigma_k$ for both reflection functors, which functor to apply should be clear from context.
  For $1\le k\le n$, write $\rep_\FF^{\langle k\rangle}(Q,D)\subset\rep_\FF(Q,D)$ for the full subcategory consisting of representations which contain no summands isomorphic to the simple representation $S_k$.
  \begin{theorem}\cite{rupel1}
    \label{th:reflection functor}
    Let $k$ be a sink or a source in $Q$ and write $(\bfX',\tilde B')=\mu_k(\bfX,\tilde B)$.  
    For any representation $V\in\rep_\FF^{\langle k\rangle}(Q,D)$ we have $X_V=X'_{\Sigma_kV}$, where $X'_{\Sigma_kV}$ denotes the quantum cluster character of $\Sigma_kV\in\rep_\FF(\mu_kQ,D)$ in the variables $\bfX'$.
  \end{theorem}

  \section{Deducing the Conjectures}
  Introduce $g$-vectors and $F$-polynomials?

  \begin{proposition}
    \label{prop:principal F-polynomials}
    Let $B$ be an $n\times n$ acyclic exchange matrix.
    Assume $k$ is a sink for $B$ and write $B'=\mu_k B$ for the mutation in direction $k$.
    Consider the tropical evaluation of the $F$-polynomials for $(B')_{prin}$ at the frozen variables corresponding to $c$-vectors of $\mu_k B_{prin}$.
    This evaluation gives the value 1 for all $F$-polynomials except for the one obtained by mutating in direction $k$.
  \end{proposition}
  \begin{proof}
    The terms of the $F$-polynomial are labeled by subrepresentations of the given rigid representation and the claim is that each of these monomials produces only non-negative exponents when evaluated at the given frozen variables.
    By definition of the $c$-vectors for $\mu_k B_{prin}$, it is only the exponent of $y_k$ that could end up being negative.
    However, observe that the total exponent is given by applying the simple reflection $s_k$ to the dimension vector of the subrepresentation.
    Since vertex 1 is a source, the simple $S_k$ cannot be a summand of any subrepresentation (assuming we were not talking about $x_{k;t_0}$).
    It follows that applying the simple reflection on dimension vectors corresponds to applying the corresponding reflection functor on quiver representations, in particular the reflected dimension vector is still a dimension vector and thus a negative exponent will never appear.
    This gives the claim.
  \end{proof}

  The following establishes \cite[Conjecture 7.17]{fomin-zelevinsky4}
  \begin{proposition}
    The denominator vector of $x_V$ is the exponent vector of the monomial obtained evaluating tropically the corresponding $F$-polynomial at $y_k^{-1}$.
  \end{proposition}
  \begin{proof}
    It is enough to observe that the $F$-polynomial has a unique monomial of maximal degree and this monomial is divisible by all other monomials.
    Indeed this is the monomial corresponding to the full subrepresentation.
  \end{proof}
  This immediately implies \cite[Conj. 6.11]{fomin-zelevinsky4} using \cite[Prop. 7.16]{fomin-zelevinsky4}.  
   

  The following, in particular, proves \cite[Conj. 2.9]{reading-stella} in the case of an acyclic initial exchange matrix.
   \begin{proposition}
    \label{prop:denominators}
    The denominator vector of $X_V$ is precisely the dimension vector $\bfv$ of $V$.
  \end{proposition}
  \begin{proof}
    The proof is identical to that of \cite[Sec. 4, Cor. 2]{caldero-keller} with appropriate modifications in the valued quiver setting, we recall the details here for convenience of the reader.  

    First note that for any representation $W$ with $[W]=\bfw$ we have $\frac{1}{d_i}\langle W,I_i\rangle=w_i=\frac{1}{d_i}\langle P_i,W\rangle$ where $I_i$ and $P_i$ denote respectively the injective hull and projective cover of the vertex simple $S_i$.  Next using the injective resolution of $S_i$ we see that $\langle W,S_i\rangle\le\langle W,I_i\rangle$ while using the projective resolution gives $\langle S_i,W\rangle\le\langle P_i,W\rangle$.  Now consider a subrepresentation $E\subset V$ with $[E]=\bfe$, applying the above considerations we see that the $i^{th}$ component of $\bfe^*+{}^*(\bfv-\bfe)$ is bounded by the $i^{th}$ component of $\bfv$:
    \[\frac{1}{d_i}\langle E,S_i\rangle+\frac{1}{d_i}\langle S_i,V/E\rangle\le\frac{1}{d_i}\langle E,I_i\rangle+\frac{1}{d_i}\langle P_i,V/E\rangle=e_i+(v_i-e_i)=v_i.\]

    To finish the proof, for each $1\le i\le n$ we must exhibit a subrepresentation $E\subset V$ which realizes this bound.  
    To construct such a subrepresentation, let $J_i$ be the set of all vertices $j$ in $Q$ for which there exists a path (possibly trivial) beginning at $i$ and ending at $j$.  
    Now set $E_j=V_j$ for $j\in J_i$ and $E_j=0$ for $j\notin J_i$.  
    Recall that in an injective resolution $0\longrightarrow S_i\longrightarrow I_i\longrightarrow I\longrightarrow 0$ the injective representation $I$ has nonzero components only at vertices $j$ which admit a nontrivial path \emph{to} vertex $i$, while in a projective resolution $0\longrightarrow P\longrightarrow P_i\longrightarrow S_i\longrightarrow 0$ the projective representation $P$ has nonzero components only at vertices $j$ which admit a nontrivial path \emph{from} vertex $i$.  
    It follows that $\langle E,S_i\rangle=\langle E,I_i\rangle-\langle E,I\rangle=\langle E,I_i\rangle$ and $\langle S_i,V/E\rangle=\langle P_i,V/E\rangle-\langle P,V/E\rangle=\langle P_i,V/E\rangle$ and thus 
    \[\frac{1}{d_i}\langle E,S_i\rangle+\frac{1}{d_i}\langle S_i,V/E\rangle=\frac{1}{d_i}\langle E,I_i\rangle+\frac{1}{d_i}\langle P_i,V/E\rangle=e_i+(v_i-e_i)=v_i\]
    as desired.
  \end{proof}

  From the fact that cluster monomials are parametrized by rigid representations we deduce \cite[Conjecture 7.6]{fomin-zelevinsky4} as follows.
  \begin{proposition}
    Suppose two cluster monomials have the same denominator vector.
    Then the cluster monomials lie in the same cluster and are equal.
  \end{proposition}
  \begin{proof}
    By Theorem~\ref{th:quantum cluster characters}, the cluster monomials are parametrized by rigid representations.
    \sayS{Update Theorem\ref{th:quantum cluster characters} to deal with cluster monomials}
    If two cluster monomials have equal denominator vectors, then the dimension vectors of the two representations they correspond to are equal.
    By rigidity, the orbits of each of these representations inside the moduli of representations with this dimension vector are both dense (see e.g.\ \cite[Corollary 2.2.5]{brion}).
    But then the representations must be isomorphic and thus correspond to the same cluster monomial.
  \end{proof}
  \begin{proposition}
    The denominator vectors of cluster variables in a seed form a $\ZZ$ basis of the root lattice
  \end{proposition}
  \begin{proof}
    Observe that, if a given cluster contains the initial cluster variable $X_i$ then the rigid representation corresponding to any other cluster variable in that cluster is supported away from $i$ \cite{some-result}.
    In particular the representations corresponding to non-initial cluster variables give a titling object for the induced subquiver and by \cite{some-other-result} a $\ZZ$-basis for the corresponding sublattice.
    Adding negative simple roots for each initial cluster variable gives the desired basis.
  \end{proof}
  Using Theorem~\ref{th:d to g} we deduce immediately from the above properties 
  \cite[Conjecture 7.10]{fomin-zelevinsky4} follows from the previous one.
  \cite[Conjecture 7.2]{fomin-zelevinsky4} follows by degree (i.e. $\bfg$-vector) considerations

  \begin{corollary}
    The mutation of the initial cluster at a sink or a source vertex transforms denominator vectors according to the simple reflection associated to that vertex.
  \end{corollary}
  \begin{proof}
    By Theorem~\ref{th:reflection functor}, the initial cluster mutation at a sink or a source vertex transforms quantum cluster characters according to the associated reflection functor on $\rep_{\FF_q}(Q,D)$.  The result then follows from \cite[Prop. 2.1]{dlab-ringel}.
  \end{proof}
  This corollary proves \cite[Conj. 2.8]{reading-stella} and hence, following \cite[Prop. 2.10]{reading-stella}, also \cite[Conj. 2.7]{reading-stella}.

  Let $E=(e_{ij})$ be the matrix given by
  \[e_{ij}=\begin{cases} -1 & \text{if $i=j$;}\\ [-b_{ij}]_+ & \text{if $i\ne j$.}\end{cases}\]
  The following result establishes \cite[Conj. 3.21]{reading-speyer}.
  \begin{theorem}
    \label{th:d to g}
    Let $\bfd_i^t\in\ZZ^n$ be a denominator vector for a cluster variable of $\cA(\bfX,\tilde B)$ and write $\bfg_i^t$ for the $g$-vector of that cluster variable.  Then $\bfg_i^t=E\bfd_i^t$.
  \end{theorem}
  \begin{proof}
    Since the $g$-vector of the cluster variable $x_\bfv$ is given by $-{}^*\bfv$, the result follows immediately from the definitions of $E$ and the operator ${}^*(-)$.
  \end{proof}

  The next result establishes \cite[Conj. 3.9, 3.10]{reading-speyer}.
  A special case of this result has appeared as \cite[Cor. 3]{caldero-keller2} for acyclic skew-symmetric cluster algebras.
  \begin{theorem}
    The subgraph of the exchange graph consisting of seeds containing any fixed collection of cluster variables is connected.
  \end{theorem}
  \begin{proof}
    The proof given in \cite{caldero-keller2} is written in terms of arbitrary hereditary abelian categories.
    In particular, their proof applies in our situation.
    More generally, one may replace the rigid indecomposable representation appearing in \cite[Section 5.4]{caldero-keller2} by an arbitrary partial tilting representation (cf. \cite[Prop. 3]{happel-rickard-schofield}).
    Then running a similar argument as in the proof of \cite[Thm. 6]{caldero-keller2} proves the general case.
    These constructions are closely related to the Iyama-Yoshino reduction of triangulated categories (cf. \cite[Sec. 4]{iyama-yoshino} or \cite[Sec. 7.2]{keller}).
  \end{proof}

  \begin{enumerate}
    \item
      \cite[Con. 7.6]{fomin-zelevinsky4}/\cite[Conj. 3.22]{reading-speyer} ($\bfd$-vectors are a basis) follows from known representation theoretic statements.
      This then implies \cite[Conj. 7.10]{fomin-zelevinsky4} using Theorem~\ref{th:d to g}.
      Then following \cite[Rem. 7.11]{fomin-zelevinsky4}, we obtain \cite[Con. 7.2]{fomin-zelevinsky4}.
    
    \item
      \cite[Conj. 7.4]{fomin-zelevinsky4} (about $\bfd$-vectors) follows easily and implies \cite[Conj. 7.5]{fomin-zelevinsky4}


    \item
      
      \begin{conjecture}[{\cite[Conjecture 4.7]{fomin-zelevinsky4}}]
        In a cluster algebra with principal coefficients any cluster is uniquely determined by its exchange matrix and $c$-vectors.
      \end{conjecture}
      \begin{proof}
        By \cite{NZ12} $g$-vectors are the the dual basis of the Langland's dual cluster algebra (which also has acylic initial seed). 
        Then use the fact that cluster monomials are parametrized by $g$-vectors.
        
        See also \cite[Conjectures 3.11 and 3.12]{reading-speyer}
      \end{proof}

    \item
      Try to prove that $\bfc$-vectors are roots following the same procedure of Najera-Chavez
  \end{enumerate}


% bibliography
\bibliographystyle{amsalpha}
\bibliography{bibliography}


\end{document}
